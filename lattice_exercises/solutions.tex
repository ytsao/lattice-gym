\documentclass[11pt,a4paper]{exam}

\usepackage{unilu}
\usepackage{wasysym}
% \usepackage[enseignants]{isup}
\usepackage{hyperref,longtable}
\usepackage{galois}
\usepackage{stmaryrd}
\usepackage{mathrsfs}
\usepackage{amssymb}
\usepackage{mathtools}
\usepackage{interval}
\usepackage{galois}
\usepackage{algpseudocode}
\usepackage{tikz-cd}

\def\titrelong{Lattice Theory for Parallel Programming\\[0.3cm]
Solutions for exercises}
\def\titrecourt{}
\def\authorsujet{}
\def\datecours{}

\date{}

\setcounter{conventions}{1}
\setcounter{pitfalls}{1}

\definecolor{ForestGreen}{RGB}{34,139,34}

\algrenewtext{EndIf}{\textbf{fi}}
\algrenewtext{EndWhile}{\textbf{done}}
\newcommand\loc[1]{{}^{\color{ForestGreen}\ell_{#1}}}
\newcommand{\expr}[1]{\mathbf{E}\llbracket #1 \rrbracket}
\newcommand{\com}[1]{\mathbf{C}\llbracket #1 \rrbracket}
\newcommand\eqdef{\,\triangleq\,}

\renewcommand{\emptyset}{\varnothing}
\newcommand{\Z}{\mathbb{Z}}
\newcommand{\R}{\mathbb{R}}
\newcommand{\N}{\mathbb{N}}
\newcommand{\Q}{\mathbb{Q}}
\newcommand{\Cl}{\mathrm{Cl}}
\newcommand{\join}{\vee}
\newcommand{\meet}{\wedge} 
\newcommand{\union}{\cup}
\newcommand{\intersection}{\cap}
\newcommand{\arrowcircle}[1][]{%
  \begin{tikzpicture}[#1]
    \draw[->] (0,0ex) -- (2em,0ex);
    \draw (1em,0ex) circle (0.7ex);
  \end{tikzpicture}%
}
\def\checked{\tikz\fill[scale=0.4](0,.35) -- (.25,0) -- (1,.7) -- (.25,.15) -- cycle;}

\newcommand{\bigmeet}{{\textstyle\bigwedge}}
\newcommand{\bigjoin}{{\textstyle\bigvee}}
\newcommand{\aleq}{\sqsubseteq}
\newcommand{\dleq}{\leq^\partial}

\lstset{
    language=C,
    basicstyle=\ttfamily\small,
    numbers=left,
    numberstyle=\tiny,
    stepnumber=1,
    showstringspaces=false,
    keywordstyle=\bfseries\color{blue},
    stringstyle=\color{red},
    commentstyle=\color{gray},
    morekeywords={include, printf} % Add additional keywords if needed
}

\graphicspath{{images/}}

\begin{document}

\section{Theoretical Exercises}

\begin{exo}
    Show that if $(X,\leq)$ is a poset, then $(X, \dleq)$ is also a poset.
    \begin{sol}
        Let $x, y, z \in X$
        \begin{enumerate}
            \item $x \dleq x \Leftrightarrow x \leq x$
            \item $x \dleq y \land y \dleq x \Leftrightarrow y \leq x \land x \leq y \implies x = y$
            \item $x \dleq y \land y \dleq z \Leftrightarrow y \leq x \land z \leq y \implies z \leq x \Leftrightarrow x \dleq z$
        \end{enumerate}
        Therefore, if $(X, \leq)$ is a poset, then $(X, \dleq)$ is also a poset.
    \end{sol}
\end{exo}

\begin{exo}
    Show that $(\mathbb{P}(X), \Rightarrow)$ is a poset.
    \begin{sol}
        Let $P, Q, R \in \mathbb{P}(X)$
        \begin{enumerate}
            \item $P \Rightarrow P \Leftrightarrow \{x \in X \mid P(x)\} \subseteq \{x \in X \mid P(x)\}$ 
            \item $P \Rightarrow Q \land Q \Rightarrow P \Leftrightarrow \{x \in X \mid P(x)\} \subseteq \{x \in X \mid Q(x)\} \land \{x \in X \mid Q(x)\} \subseteq \{x \in X \mid P(x)\} \Leftrightarrow \{x \in X \mid P(x)\} = \{x \in X \mid Q(x)\} \Leftrightarrow P = Q$
            \item $P \Leftarrow Q \land Q \Leftarrow R \Leftrightarrow \{x \in X \mid P(x)\} \subseteq \{x \in X \mid Q(x)\} \land \{x \in X \mid Q(x)\} \subseteq \{x \in X \mid R(x)\} \Leftarrow \{x \in X \mid P(x) \} \subseteq \{x \in X \mid R(x)\} \Leftrightarrow P \Leftarrow Q$
        \end{enumerate}
    \end{sol}
\end{exo}

\begin{exo}
    If $(X, \leq_X)$ and $(Y, \leq_Y)$ are chains, then their linear sum $(X \oplus Y, \leq)$ is also a chain.

    \begin{sol}
        \begin{tikzcd}[row sep=large, column sep=large]
          & 1 \\
          & 0 \arrow[u, -, line width=0.5pt] \\
          & \{a,b\} \arrow[u, -] \\ 
          \{a\} \arrow[ur, -] & & \{b\} \arrow[ul, -] \\ 
          & {\emptyset} \arrow[ul, -] \arrow[ur, -]
        \end{tikzcd}
    \end{sol}
\end{exo}

\begin{exo}
    If $(X, \leq_X)$ and $(Y, \leq_Y)$ are chains, then their linear sum $(X \oplus Y, \leq)$ is also a chain.

    \begin{sol}
        Since $(X, \leq_X)$ and $(Y, \leq_Y)$ are chains, it means that all elements in each of which are comparable.\\
        Also, by linear sum, all elements of $X$ have to be placed below all elements of $Y$, while preserving the original orders within $X$ and $Y$.
        Therefore, $\forall x \in X, \forall y \in Y, x \leq y$.\\
        So, $(X \oplus Y, \leq)$ is a chain.
    \end{sol}
\end{exo}

\begin{exo}
    The pointwise order of two posets is an order.

    \begin{sol}
        Let $(X, \leq_X)$ and $(Y, \leq_Y)$ be two posets. 
        Let $(x_1, y_1), (x_2, y_2), (x_3, y_3) \in X \times Y$.
        \begin{enumerate}
            \item $(x_1, y_1) \leq (x_1, y_1) \Leftrightarrow x_1 \leq_X x_1 \land y_1 \leq_Y y_1$
            \item $(x_1, y_1) \leq (x_2, y_2) \land (x_2, y_2) \leq (x_1, y_1) \Leftrightarrow (x_1 \leq_X x_2 \land y_1 \leq_Y y_2) \land (x_2 \leq_X x_1 \land y_2 \leq_Y y_1) \Leftrightarrow x_1 = x_2 \land y_1 = y_2 \Leftrightarrow (x_1, y_1) = (x_2, y_2)$
            \item $(x_1, y_1) \leq (x_2, y_2) \land (x_2, y_2) \land (x_3, y_3) \Leftrightarrow (x_1 \leq_X x_2 \land y_1 \leq_Y y_2) \land (x_2 \leq_X x_3 \land y_2 \leq_Y y_3) \implies x_1 \leq_X x_3 \land y_1 \leq_Y y_3 \Leftrightarrow (x_1, y_1) \leq (x_3, y_3)$   
        \end{enumerate}
    \end{sol}
\end{exo}

\begin{exo}
    What is the Hase diagram of $B \times A$ with lexicographic order?

    \begin{sol}
        $B \times A = \{(0,a), (0,b), (0,c), (1,a), (1,b), (1,c)\}$\\
        \begin{tikzcd}[row sep=large, column sep=large]
            (1,b) & & (1,c)\\
            & (1,a) \arrow[ul,-,line width=0.5pt] \arrow[ur,-,line width=0.5pt]\\
            (0,b) \arrow[ur,-] & & (0,c) \arrow[ul,-]\\
            & (0,a) \arrow[ul,-] \arrow[ur,-]
        \end{tikzcd}    
    \end{sol}
\end{exo}

\begin{exo}
    If $(X, \leq_X)$ and $(Y, \leq_Y)$ are chains, then the lexicographic order on product is also a chain. However, this is not necessarily ture for the pointwise order on the product.

    \begin{sol}
        By definition of lexicographic order. If $(X, \leq_X)$ and $(Y, \leq_Y)$ are chains, then all elements in their product are comparable.
        Because it only depends on either $\leq_X$ or $\leq_Y$. However, in pointwise order, the elements in the product of $(X, \leq_X)$ and $(Y, \leq_Y)$ might be incomparable.\\ 
        For example, $X = \{1, 2\}, Y = \{a,b,c\}$, their order is defined as natural and alphabetic. \\
        Hence $X \times Y = \{(1,a),(1,b),(1,c),(2,a),(2,b),(2,c)\}$, where $(1,b)$ and $(2,a)$ are incomparable.\\ 
        Since $(1,b) \nleq (2,a), 1 \leq_X 2,$ but $b \nleq a$, $(X \times Y, \leq)$ is not a chain.        
    \end{sol}
\end{exo}

\begin{exo}
    Let $(X, \leq)$ and $(Y, \leq)$ be posets, and let $f \colon X \rightarrow Y$ be a function. 
    The following conditions are equivalent. 
    \begin{enumerate}
        \item $f$ is an order isomorphism.
        \item $f$ is a monotone bijective map and the map $f^{-1}$ is monotone.
    \end{enumerate}

    \begin{sol}
        \begin{itemize}
            \item Assume (i) is correct. So $f$ is an onto order embedding. It means that:
            \[
                x_1 \leq x_2 \Leftrightarrow f(x_1) \leq f(x_2)
            \]
            which is a monotone bijective map.
            $f^{-1} \colon Y \rightarrow X,$ by (i), $f$ is an onto function.
            Thus, every element in $Y$ can be mapped by at least one element in $X$ with $f(x)$.
            Also, $f$ is one-to-one map, it means that 1 element in $X$ will map to exactly one element in $Y$.
            By considering 2 properties, $\forall x_1, x_2 \in X, f^{-1}(f(x_1)) \text{ and } f^{-1}(f(x_2))$ will map back to $x_1 \text{ and } x_2,$ respectively.

            \item Assume (ii) is correct. Because $f$ is a monotone bijective map, it is order-embedding.
            Also, when $f^{-1}$ is  monotone, it implies that every element in $Y$ can find the corresponding element in $X$ meaning $f^{-1}$ is an onto. Therefore, by combing order-embedding and onto, we can conclude that $f$ is order-isomorphism.
        \end{itemize} 
    \end{sol}
\end{exo}

\begin{exo}
    Prove that the following statements are true.
    \begin{enumerate}
        \item The function $f \colon (\N, \leq) \rightarrow (\N,\leq)$ defined by $f(n) = 2n$ is order-preserving.
        \item The function $g \colon (\R, \leq) \rightarrow (\R, \leq)$ defined by $g(x) = x^2$ is not order-preserving. For instance, $-2 < -1$, but $g(-2) = 4 > 1 = g(-1)$.
        \item The inclusion map $i \colon (\N, \leq) \rightarrow (\Z, \leq)$ defined by $i(n) = n$ is an order-embedding.
        \item The function $f \colon (\N, \leq) \rightarrow (\N, \leq)$ defined by $f(n) = 2n$ is an order embedding.
        \item The function $f \colon (\R, \leq) \rightarrow (\R, \leq)$ defined by $f(x) = \lfloor x \rfloor$ is order-preserving but is not an order embedding.
        \item The function $f \colon (\N, \leq) \rightarrow (\N^{*}, \leq)$ defined by $f(n) = n + 1$ is an order isomorphism.
        \item The function $h \colon (\N \times \N, \leq) \rightarrow (\N, \leq)$, where $\N \times \N$ is equipped with the pointwise order, defined by $f(x,y) = x + y$ is order-preserving but not an embedding.
        \item If $(X_1, \leq)$ and $(X_2, \leq)$ are two posets that the projections maps $\pi_1, \pi_2 \colon X_1 \times X_2 \rightarrow X_i$ defined as $\pi_i(x_1,x_2) = x_i$ is order-preserving if $X_1 \times X_2$ is equipped with the pointwise order.
    \end{enumerate}

    \begin{sol}
        (1)-(4) and (6) are trivial.
        \begin{enumerate}
            \item Let $f(x) = 1, f(y) = 1,$ then: 
            $f(x) \leq f(y),$ but $(x,y)$ can be $(1.5, 1.3)$ which means $x \nleq y$. So, $f$ is not order-embedding.
            \item $f$ is a monotone function. By pointwise-order, when $(x_1,y_1) \overset{\centerdot}{\leq} (x_2, y_2) \Leftrightarrow x_1 \leq x_2 \land y_1 \leq y_2$. So, $x_1+y_1 \leq x_2+y_2 \Leftrightarrow f((x_1,y_1))\leq f((x_2,y_2))$. $f$ is not order-embedding function. If $f((x_1,y_1)) \leq f((x_2,y_2)) \equiv 3 \leq 3$ $(x_1, y_1)$ can be $(2,1)$ and $(x_2,y_2)$ can be $(1,2)$. In such a case, $(x_1, y_1) \nleq (x_2,y_2)$, so it is not order-embedding.
            \item $\forall x_{11}, x_{12} \in X_1, \forall x_{21}, x_{22} \in X_2$, $(x_{11},x_{12}) \leq (x_{12}, x_{22}) \Leftrightarrow x_{11} \leq x_{12} \land x_{21} \leq x_{22}$. So, by $\pi_i,$ it will reserve one of dimensions, it means that the order relation will reserve the corresponding one. \\
            Therefore, 
            $\begin{cases}
               (x_{11}, x_{12}) \leq (x_{12}, x_{22}) \Rightarrow \pi_1((x_{11},x_{12})) \leq \pi_1((x_{12},x_{22})) &\Leftrightarrow x_{11} \leq x_{12} \\
               (x_{11}, x_{12}) \leq (x_{12}, x_{22}) \Rightarrow \pi_2((x_{11},x_{12})) \leq \pi_2((x_{12},x_{22})) &\Leftrightarrow x_{21} \leq x_{22}
            \end{cases}$.\\
            For lexicographic order, $\pi_1$ is order-preserving $\Leftrightarrow x_{21} \leq x_{22}$. Yet, $\pi_2$ is not.
        \end{enumerate}        
    \end{sol}
\end{exo}

\begin{exo}
    Let $X = \{a,b,c,d\}$. Prove that the powerset poset $(2^X, \subseteq)$ is isomorphic to the predicate post $(\mathbb{P}, \Rightarrow)$. Then, prove that the previous statement hods for any set $X$ (even an infinite one).

    \begin{sol}
        Define $\phi \colon 2^X \rightarrow \mathbb{P}(X)$. Let $g \in 2^X, \phi(y) = \{x \in X \mid x \in y\}$.\\
        By this definition, $\phi$ is order-isomorphism, bijective. Therefore, $(2^X, \subseteq) \overset{~}{=} (\mathbb{P}(X), \Rightarrow)$ 
    \end{sol}
\end{exo}

\begin{exo}
    Prove that $(\Z, \leq)$ is isomorphic to $(\Z, \leq^\partial)$. Is $(\N,\leq)$ isomorphic to $(\N, \leq^\partial)$?

    \begin{sol}
        We define $f \colon (\Z, \leq) \rightarrow (\Z, \dleq), f(x) \triangleq -x,$ such that $x_1 \leq x_2 \Leftrightarrow f(x_1) \dleq f(x_2)$
        Therefore, $(\Z, \leq) \overset{~}{=} (\Z, \dleq)$.\\
        Because, in $(\N, \leq)$, there is a bottom element, but top element. While, in $(\N, \dleq)$, there is no bottom element but top element.
        So, we cannot find an order-isomorphism between them, they're not isomorphic. 
    \end{sol}
\end{exo}

\begin{exo}
    Characterize the $n \in \N$ whose divisor poset is isomorphic to $(2^{\{0,1\}}, \subset)$.

    \begin{sol}
        Prime number
    \end{sol}
\end{exo}

\begin{exo}
    Prove that $(\Z,\leq)$ is isomorphic to $(\N, \leq^\partial) \oplus (\N,\leq)$.

    \begin{sol}
        Define $f \colon \Z \rightarrow \N$\\ 
        $f \triangleq   \begin{cases}
                            x-1,    &\text{ if } x > 0 \\
                            0,      &\text{ if } x = 0 \\ 
                            \vert x \vert, &\text{ if } x < 0
                        \end{cases}$ 
    \end{sol}
\end{exo}

\begin{exo}
    Let $C$ be the set of subset $X$ of $\N$ such that $\N \setminus X$ is finite. Show that $C$ is a filter in $(2^\N, \subseteq)$.

    \begin{sol}
        Let $X \in C, Y \supseteq X$ such that $\N \setminus X$ is finite (by given condition). Then $\N \setminus X \supseteq \N \setminus Y$, therefore, $\N \setminus Y$ is also finite. It means that $Y \in C$. Hence, $C$ is an up-set (upward-closed). In addition, for any $X, Y \in C$, there is $Z \in C$ such that $Z \subseteq X \land Z \subseteq Y$. This $Z$ would be $\{ \infty \}$. Assume that $X \land Y = Z, X \supseteq Z \land Y \supseteq Z. \N \setminus X \text{ is finite }, \N \setminus Y \text{ is finite }, \N \setminus Z = \N \setminus (X \cap Y) = \N \setminus X \cup \N \setminus Y$. Therefore, $C = \{X \in 2^{\N} \mid \{\infty\} \subseteq X \}$.        
    \end{sol}
\end{exo}

\begin{exo}
    Let $(X, \leq)$ be a poset and $Q \subseteq X$. We set 
    \begin{enumerate}
        \item Show that $\downarrow Q$ is an down-set that contains Q. Deduce that $\uparrow Q$ is an up-et that contains Q.
        \item Let $(Up(X), \supseteq)$ be the set of up-sets of $(X, \leq)$ ordered by reverse inclusion. Show that the map $f \colon X \rightarrow Up(X)$ defined as $f(x) = \uparrow x$ is an order-embedding.
    \end{enumerate}

    \begin{sol}
        $\downarrow Q = \{x \in X \mid \exists q \in Q x \leq q\}$, by definition. Down-set means that if $y \in Y \land x \leq y,$ then $x \in Y,$ where $Y \subseteq X$. Since $Q \subseteq X \land Q \neq \emptyset,$ meaning that $(Q, \leq)$ is a poset as well.
        By definition of $\downarrow Q$, since $(X, \leq) \& (Q, \leq)$ are posets, $\forall x \in X \exists q \in Q, x \leq q, x \text{ can be } q$ (reflexive) such that $\downarrow Q$ contains $Q$. Moreover, $\downarrow Q$ contains every element that is lower than at least one element in $Q$. It means that if we have an element $x \in X$ and an element $y \in \downarrow Q$ and their relation is $x \leq y$. Then it implies $x$ must in $\downarrow Q$, by the definition of $\downarrow Q$. Therefore, $\downarrow Q$ is a down-set. Dually, $\uparrow Q$ is an up-set that contains $Q$.\\ 
        \begin{itemize}
            \item Show that the map $f \colon X \rightarrow UP(X)$ defined as $f(x) = \uparrow x$ is an order-embedding.
            \item Order embedding: $x \leq y \Leftrightarrow f(x) \supseteq f(y)$.
            \item $\uparrow x = \{y \in X \mid y \geq x\}$.
        \end{itemize} 
        $(\Rightarrow)$ \\
        Since $x \leq y, \vert f(x) \vert \geq \vert f(y) \vert$. Also, by the definition of $\uparrow x$. $\uparrow x$ constains every element that above $x$ (including $y$, and the elements above $y$). But $x$ is below $y$, it means that $\uparrow y$ does not contain $x$. Still $\uparrow y$ contains every element that $y$. Hence $\uparrow y$ is a subset of $\uparrow x$ such that $\uparrow x \supseteq \uparrow y \Leftrightarrow f(x) \subseteq f(y)$. \\ 
        $(\Leftarrow)$ \\ 
        When $f(x) \supseteq f(y) \Leftrightarrow \uparrow x \supset \uparrow y$ is true. We can find the minimal elements $x, y \text{ in } f(x), f(y),$ respectively. Both elements are in $X$ such that $x \leq y$. Therefore, the map $f$ is an order-embedding, it is an injective function. 
    \end{sol}
\end{exo}

\begin{exo}
    Prove that if $(X, \leq)$ has a top element, then it is unique. Similarly, if $(X, \leq)$ has a bottom element, then it is unique.

    \begin{sol}
        Suppose that we have 2 top elements, $x, y$, in $(X, \leq)$. By the definition of top element, $x \& y$ are the upper bounds of $X$. By the definition of upper bound, we know that $x \leq \land y \leq x$. By antisymmetric, we can therefore conclude $x = y$. Therefore, if $(X, \leq)$ has a top element, it is unique. Dually, if $(X, \leq)$ has a bottom element, it is also unique.        
    \end{sol}
\end{exo}

\begin{exo}
    Prove that in any lattice $(L,\leq)$, for all $x,y,z \in L \colon$
    \begin{enumerate}
        \item $x \join y = y \join x$ and $x \meet y = y \meet x$ (commutativity)
        \item $(x \join y) \join z = x \join (y \join z)$ and $(x \meet y) \meet z = x \meet (y \meet z)$ (associativity)
        \item $x \join x = x$ and $x \meet x = x$ (idempotence)
        \item $x \join (x \meet y) = x$ and $x \meet (x \join y) = x$ (absorption)
    \end{enumerate}
    
    \begin{sol}
        \begin{enumerate}
            \item $x \lor y = \{x,y\}^u = y \lor x$
            \item $(x \lor y) \lor x = \{x,y\}^u \lor z = \{\{x,y\}^u, z\}^u = \{x,y,z\}^u$, $x \lor (y \lor z) = x \lor \{y,z\}^u = \{x,\{y,z\}^u\}^u=\{x,y,z\}^u$
            \item $x \lor x = \{x, x\}^u$
            \item $x \land y \leq x \Leftrightarrow x \lor (x \land y) \leq x \lor x \Leftrightarrow x \lor (x \land y) \leq x$. By join definition, $x \leq x \lor (x \land y)$. Therefore, $x \lor (x \land y) = x$.
        \end{enumerate} 
    \end{sol}
\end{exo}

\begin{exo}
    Prove that if $(L, \join, \meet, 0, 1)$ is an algebraic bounded lattice, then in the corresponding order-theoretic lattice $(L, \leq)$, the element $0$ is the bottom element and $1$ is the top element.

    \begin{sol}
        By the definition of bounded lattice algebra $\mathcal{L} = (L, \join, \meet, 0, 1)$. 
        \begin{enumerate}
            \item $\forall x \in L, x \join 0 = x, \text{ and } x \meet 1 = x$.
            \item $\forall x \in L, x \meet 0 = 0, \text{ and } x \join 1 = 1$.
        \end{enumerate}
        By connecting lemma, we know that when $x \join 0 = x, x \meet 0 = 0$, it implies that $0 \leq x$, since every element $x \in L$ is greater than 0. In other words, $0$ is a lower bound for all $x \in L$. Therefore, $0$ is the bottom element in $L$.
        Similarly, when $x \join 1 = 1, x \meet 1 = x$, it means that $x \leq 1$, since every $x \in L$ is lower than $1$. It is saying that $1$ is an upper bound for all $x \in L$. Therefore, $1$ is the top element in $L$.
    \end{sol}
\end{exo}

\begin{exo}
    Consider the following statements about lattice operations and constructions:
    \begin{enumerate}
        \item The disjoint union of lattices is a lattice.
        \item The linear sum of lattices is a lattice.
        \item The lexicographic order on the product of lattices might not be a lattice.
        \item The pointwise order on the product of lattices is always a lattice, with join and meet operations computed pointwise.
    \end{enumerate}
    For each statement, provide a proof or a counterexample to justify why the statement is true or false.

    \begin{sol}
        \begin{enumerate}
            \item false, when $x \in P, y \in Q, x \& y$ are incomparable. Also, meaning $x \& y$ do not have the least $ub$ and the greatest $lb$.
            \item true, by definition of linear sum, every element can always find $lub \& glb$.
            \item true, in lexicographic order, the second dimension might not comparable such that it might not be a lattice. So $lub \& glb$ are defined by pointwise join/meet. For example, $\mathbb{M}_3 \times \Z, lub \& glb$ do not exist! 
            \item true.
        \end{enumerate}
    \end{sol}
\end{exo}

\begin{exo}
    If $(L, \meet, \join)$ is a lattice and $S$ is a sublattice of $L$, then $(S, \join, \meet)$ is a lattice.

    \begin{sol}
        When $S$ is a sublattice of $L$, by definition, $\forall x, y \in S, x \join y \in S, x \meet y \in S$. It implies that every pair of element in $S$ has the least upper bound and the greatest lower bound which are defined in $S$. Therefore, $S$ itself is a lattice.
    \end{sol}
\end{exo}

\begin{exo}
    Let $(L, \join, \meet, 0, 1)$ be a bounded lattice and $S$ be a sublattice of $L$.
    \begin{enumerate}
        \item Prove that if $S$ is a 0-sublattice, then $0$ is the bottom element of $S$.
        \item Prove that if $S$ is a 1-sublattice, then $1$ is the top element of $S$.
        \item Give an example of a sublattice that has a bottom element different from $0$ and a top element different from $1$.
    \end{enumerate}

    \begin{sol}
        (1) \& (2) are similar to exercise 63. \\ 
        (3) \\ 
        Let $L$ is a bounded chain, $\{0, 1, 2, 3\}$, where top is $3$ and bottom is $0$. There exists a sublattice $S = \{1, 2\}$. $1$ is the bottom of $S$, but not $0-$element in $L$, $2$ is the top of $S$, but not $1-$element in $L$.
    \end{sol}
\end{exo}

\begin{exo}
    Let $f \colon L_1 \rightarrow L_2$ be a lattice isomorphism. Prove that $f^{-1}$ is a lattice isomorphism.

    \begin{sol}
        $f^{-1} \colon L_2 \rightarrow L_1$. Let $x_1, y_1 \in L_1, x_2, y_2 \in L_2$. We know $f(x_1 \join y_1) = f(x_1) \join f(y_1), f(x_1 \meet y_1) = f(x_1) \meet f(y_1)$. Also, $f$ is bijective and monotone. We have to show that 
        \begin{enumerate}
            \item $f^{-1}$ is lattice homomorphism
            \item $f^{-1}$ is bijective
        \end{enumerate}
        $[\text{join}]$\\
        $f(x_1) \join f(y_1) = f(x_1 \join y_1) \Leftrightarrow f^{-1}(f(x_1))\join f^{-1}(f(y_1)) = f^{-1}(f(x_1 \join y_1)) \Leftrightarrow f^{-1}(f(x_1))\join f^{-1}(f(y_1)) = f^{-1}(f(x_1) \join f(y_1))$. \\
        $[\text{meet}]$ \\
        $f(x_1) \meet f(y_1) = f(x_1 \meet y_1) \Leftrightarrow f^{-1}(f(x_1))\meet f^{-1}(f(y_1)) = f^{-1}(f(x_1 \meet y_1)) \Leftrightarrow f^{-1}(f(x_1)) \meet f^{-1}(f(y_1)) = f^{-1}(f(x_1) \meet f(y_1))$.\\
        Since $f$ is bijective, $f^{-1}$ will be bijective as well.
    \end{sol}
\end{exo}

\begin{exo}
    Prove that if $f \colon L_1 \rightarrow L_2$ is a bounded lattice isomorphism, then $f^{-1} \colon L_2 \rightarrow L_1$ is also a bounded lattice isomorphism. 

    \begin{sol}
        Given that $f \colon L_1 \rightarrow L_2$ is a bounded lattice isomorphism. So, we know $f$ is bounded lattice isomorphism and bijective. Hence, let $x_1, y_1 \in L_1, x_2, y_2 \in L_2$. \\
        $[\text{join}]$\\ 
        $f(x_1 \join y_1) = f(x_1) \join f(y_1) \Leftrightarrow f^{-1}(f(x_1 \join y_1)) = f^{-1}(f(x_1)) \join f^{-1}(f(y_1)) \Leftrightarrow f^{-1}(f(x_1) \join f(y_1)) = f^{-1}(f(x_1)) \join f^{-1}(f(y_1)) \Leftrightarrow f^{-1}(x_2 \join y_2) = f^{-1}(x_2) \join f^{-1}(y_2)$\\
        $[\text{meet}]$\\
        $f(x_1 \meet y_1) = f(x_1) \meet f(y_1) \Leftrightarrow f^{-1}(f(x_1 \meet y_1)) = f^{-1}(f(x_1)) \meet f^{-1}(f(y_1)) \Leftrightarrow f^{-1}(f(x_1) \meet f(y_1)) = f^{-1}(f(x_1)) \meet f^{-1}(f(y_1)) \Leftrightarrow f^{-1}(x_2 \meet y_2) = f^{-1}(x_2) \meet f^{-1}(y_2)$\\
        Since $f$ is bijective, $f^{-1}$ is bijective as well.
    \end{sol}
\end{exo}

\begin{exo}
    Give an example of bounded lattices $L_1$ and $L_2$ and lattice homomorphism $f \colon L_1 \rightarrow L_2$ which is not a bounded lattice homomorphism.

    \begin{sol}
        Let $L_1 = \{\bot, a, b, \top\},$ where $a || b$. Let $L_2 = \{0, 1, 2, 3\}$, where its order is defined as natural. $f$ is defined as $f(\bot) = f(a) = 1, f(b) = f(\top) = 2$. In this case, two tops and bottoms from $L_1$ and $L_2$ are not mapped.
    \end{sol}
\end{exo}

\begin{exo}
    Give a precise inductive definition of the interpretation of a (bounde) lattice term on a (bounded) lattice.

    \begin{sol}
        It is very similar with the operations used in $L$ and the elements shown in $L$.
        \begin{enumerate}
            \item $\forall x \in X, t^L(x) = x$
            \item $t^L(0) = 0, t^L(1) = 1$
            \item $t(x,y) = x \join y, t^L(x,y) = x \join y. t(x,y) = x \meet y, t^L(x,y) = x \meet y$
        \end{enumerate}
    \end{sol}
\end{exo}

\begin{exo}
    Any lattice satisfies the equation $(x \join y) \join z = x \join (y \join z)$. If $X = \{a, b\}$ then the lattice $(2^X, \union, \intersection)$ satisfies the equation $x \join (y \meet z) = (x \join y) \meet (x \join z)$ while the lattice $N_5$ depicted Fig. 2 doesn't satisfy it.

    \begin{sol}
        $a \join (b \meet c) = a \neq (a \join b) \meet (a \join c) = c$
    \end{sol}
\end{exo}

\begin{exo}
    Let $\mathcal{K}$ be a class of (bounded) lattices and $t = s$ be a class of (bounded) lattice equations. Show that if every element of $\mathcal{K}$ satisfies $t = s$, then every element of $\mathbb{S}(\mathcal{K}), \mathbb{P}(\mathcal{K}) \text{ and } \mathbb{H}(\mathcal{K})$ satisfies $t = s$.

    \begin{sol}
        $\forall L \in K \colon t^L(\overset{\rightarrow}{a} = s^L(\overset{\rightarrow}{a})),$ where $\overset{\rightarrow}{a} \in L$.
        \begin{itemize}
            \item sublattices: Let $subL$ as the set of some sublattices of $L$. $\forall SL \in subL \colon t^{SL}(\overset{\rightarrow}{b}) = t^L(\overset{\rightarrow}{b} = s^L(\overset{\rightarrow}{b}) = s^{SL}(\overset{\rightarrow}{b}))$. Therefore, $SL \in L,$ meaning that to interpret $t^{SL}(\overset{\rightarrow}{b}),$ we can use $t^{L}(\overset{\rightarrow}{b})$. Similar for $s$.
            \item Homomorphism image: Let $L_1, L_2 \in K, f \colon L_1 \rightarrow L_2$ is surjective homomorphism. $f(t^{L_1}(\overset{\rightarrow}{a})) = t^{L_2}(f(a_1), \dots, f(a_n))$. $t^{L_2}(f(\overset{\rightarrow}{a})) = f(s^{L_1}(\overset{\rightarrow}{a})) = s^{L_2}(\overset{\rightarrow}{a}).$
            \item Products: Let $(L_i)_{i \in I}$ be a family of lattices. $\pi_{i\in I}L_i$ is the set of all tuples $(a_i)_{i\in I}$ with $a_i \in L_i,$ equipped with pointwise operators. $t^L(\overset{\rightarrow}{a}) = (t^{L_i}(\overset{\rightarrow}{a}_i))_i = (s^{L_i}(\overset{\rightarrow}{a}_i)_i) = s^L(\overset{\rightarrow}{a})$
        \end{itemize}
    \end{sol}
\end{exo}

\begin{exo}
    Give an example of a non-distributive lattice whose every element has a complement, but there is a at least one element which has tow complements.

    \begin{sol}
        $\mathbb{M}_3 \text{ or } \N_5$
    \end{sol}
\end{exo}

\begin{exo}
    Show that the class of complemented bounded distributive lattices is not equational.

    \begin{sol}
        This is a complemental bounded distributive lattices. Yet, its sublattices are not, e.g., $\{\bot, a, \top\}$. In general, any chain in this lattice is not satisified. Therefore, $a$ does not have a complement.
    \end{sol}
\end{exo}

\begin{exo}
    Prove that in a Boolean algebra, De Morgan's laws hold:
    \[
        (x \join y)^\prime = x^\prime \meet y^\prime \text{ and } (x \meet y)^\prime = x^\prime \join y^\prime
    \]

    \begin{sol}
        $(x \join y) \join (x^\prime \meet y^\prime) = (x \join y \join x^\prime) \meet (x \join y \join y^\prime) = (1 \join y) \meet (x \join 1) = 1 \meet 1 = 1$ \\ 
        $(x \meet y) \meet (x^\prime \join y^\prime) = (x \meet y \meet x^\prime) \join (x \meet y \meet y^\prime) = (0 \meet y) \join (x \meet 0) = 0 \join 0 = 0$
    \end{sol}
\end{exo}

\begin{exo}
    Prove that in a Boolean algebra $B$, the complementation operation is a bounded lattice isomorphism between the underlying bounded lattice of $B$ and its order dual. (\textit{Hint:} Prove that (i) $B^\partial$ equipped with the complementation operation from B is a Boolean algebra, (ii) the complementation operation $\prime \colon B \rightarrow B^\partial$ is a monotone map which is its own inverse.)
    
    \begin{sol}
        
    \end{sol}
\end{exo}

\begin{exo}
    Let B be the set of subsets $X$ of $\N$ that satisfy B is finite or $\N \setminus B$ is finite. Show that B is a Boolean algebra.

    \begin{sol}
        
    \end{sol}
\end{exo}

\begin{exo}
    We adopt the notation of the previous paragraphs.
    \begin{enumerate}
        \item Show that $\downarrow f$ is an ideal in $X \arrowcircle Y$.
        \item Show that $\downarrow f$ is a complete bounded lattice in its own right, with $\emptyset$ and $f$ as bottom and tope element, respectively.
    \end{enumerate}

    \begin{sol}
        
    \end{sol}
\end{exo}

\begin{exo}
    Let $c \colon 2^{\R^\infty} \rightarrow 2^{\R^\infty}$ be defined as $c(X) = (\bigjoin X]$.
    \begin{enumerate}
        \item Show that $c$ is a closure operator.
        \item Describe the set K of closed elements. 
        \item Describe the complete lattice operation on K.
        \item Describe $c(x)$ in terms of closed elements for every $x \in 2^{\R^\infty}$.
    \end{enumerate}

    \begin{sol}
        
    \end{sol}
\end{exo}

\begin{exo}
    Let $(L, \meet, \join, \bot, \top)$ be a bounded lattice, and $X$ be an infinite set of variables. For any $n \geq 0$, let $T_n(X)$ the set of $n-$ary bounded lattice terms over $X$ and set $T(X) \coloneqq \union_{n \geq 0}T_n(X)$. Define $c \colon 2^L \rightarrow 2^L$ by 
    \[
        c(A) = \{t^L(a_1,...,a_n) \mid n \geq 0, t \in T_n(X) \text{ and } a_1,...,a_n \in A\}.
    \]
    \begin{enumerate}
        \item Show that $c$ is a closure operator.
        \item Describe the set K of closed elements.
        \item Describe the complete lattice operations on K. 
        \item Describe $c(A)$ in terms of closed elements for every $A \in 2^L$.
    \end{enumerate}

    \begin{sol}
        
    \end{sol}
\end{exo}

\begin{exo}
    Show that the constructions of Table 1 define are mutually inverse.

    \begin{sol}
        
    \end{sol}
\end{exo}

\begin{exo}
    Let $(X, \leq)$ be a poset. Define $U \coloneqq \{\uparrow x \mid x \in X\}$ and $D \coloneqq \{\downarrow x \mid x \in X\}$, as well as the maps $\alpha \colon D \rightarrow U$ and $\gamma \colon U \rightarrow D$ defined as $\gamma(\uparrow x) \coloneqq \downarrow x$ and $\alpha(\downarrow x) \coloneqq \uparrow x$. Prove that 
    \[
       (L, \leq) \galois{\alpha}{\gamma} (M, \aleq)
    \]

    \begin{sol}
        
    \end{sol}
\end{exo}

\begin{exo}
    Let $(C, \leq)$ be a complete totally ordered set and $\rightarrow$ be the binary operation defined on C by $a \rightarrow b \coloneqq \bigjoin\{x \in C \mid a \meet x \leq b\}$. Prove that for any $a \in C$, we have 

    \begin{sol}
        
    \end{sol}
\end{exo}

\begin{exo}
    (From Galois Connection to Closure Operator). Assume that 

    \begin{sol}
        
    \end{sol}
\end{exo}

\begin{exo}
    Identify the Galois connections among the Galois connections we have introduced so far.

    \begin{sol}
        
    \end{sol}
\end{exo}

\end{document}